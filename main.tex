\documentclass{article}
\usepackage{amsmath}
\usepackage{mathtools}
\usepackage{graphicx}% Include figure files
\usepackage{bm}% bold math
\usepackage{verbatim}
\usepackage{color}


\newcommand{\warning}[1]{{\textsf{{\textcolor{red}{{[#1]}{}}}}}}
\let\vec\mathbf

\begin{document}

\title{A cumulant analysis of free energy decompositions}
\author{Jason A. Wagoner and Justin L. MacCallum}

\date{ \today}

\maketitle

\begin{abstract}
Stuff...
\end{abstract}

\section{Introduction}

\subsection{Free energy decompositions offer insight into driving forces and mechanisms.}

Why are these things useful? Give examples where they give insight.




\subsection{The interpretation of free energy decompositions faces two challenges.}

Explain non-additivity. Essentially, changes to one Hamiltonian component will affect the distribution of conformations, which, in turn, affects all of the other Hamiltonian components. This is a fundamental issue inherent to the concept of free energy, and is not due to some artifact of free energy calculations. It happens in both experiment and simulation. This will always be an issue, but decompositions can still be useful for understanding driving forces, e.g. utility of alanine mutagenesis, etc. Care is required in interpretation, but these measurements are clearly valuable.

Explain path-dependence. When different ways of doing a calculation give different answers, interpretation is difficult or impossible. However, we will show that this issue can be mitigated through specific analysis protocols.




\section{Theory}

\subsection{Free energy changes can be calculated by thermodynamic integration.}

Consider a system with Hamiltonian $H(x, \lambda) = H_1(x, \lambda) + H_2(x, \lambda) + \ldots$ parameterized by a control parameter $\lambda$. The components ($H_1, H_2, \ldots$) each depend on $\lambda$ in some arbitrary way. For ease of presentation, we largely focus on the case of two components, but our results readily generalize to an arbitrary number of Hamiltonian components. Hereafter, to simplify notation, we omit the $x$-dependence of the Hamiltonian and its components.

The excess free energy compared to an ideal gas is:
\begin{equation}\label{eq:dA}
\Delta A(\lambda) = -\beta^{-1} \ln \int e^{-\beta H(\lambda)} dx,
\end{equation}
where $\beta=1/k_BT$, $k_B$ is Boltzmann's constant, and $T$ is the absolute temperature.

We can compute the free energy difference between the states $\lambda=0$ and $\lambda=1$ using thermodynamic integration:
\begin{align}
\Delta\Delta A =& \Delta A(1) - \Delta A(0) \nonumber\\
               =& \int_0^1 \frac{d}{d\lambda} \Delta A(\lambda) d\lambda \nonumber\\
               =& \int_0^1 \left\langle \frac{dH(\lambda)}{d\lambda}\right\rangle_\lambda 
                d\lambda \label{eq:TI},
\end{align}
where the angle brackets denote the ensemble average:
\begin{equation}
\left\langle f \right\rangle_\lambda = \frac
	{\int f(x, \lambda) e^{-\beta H(\lambda)} dx}
    {\int e^{-\beta H(\lambda)} dx}.
\end{equation}




\subsection{The most obvious free energy decomposition is path-dependent.}

The form of Eq.~\ref{eq:TI} suggests a natural decomposition of the free energy:
\begin{align}
\Delta\Delta A =& \Delta\Delta A_1 + \Delta\Delta A_2 \nonumber\\
\Delta\Delta A_1 =&
	\int_0^1 \left\langle \frac{dH_1(\lambda)}{d\lambda}\right\rangle_\lambda d\lambda \nonumber\\
\Delta\Delta A_2 =&
	\int_0^1 \left\langle \frac{dH_2(\lambda)}{d\lambda}\right\rangle_\lambda d\lambda.    
\label{eq:naive}
\end{align}
However, it is known that the free energy components obtained in this way are path-dependent. $\Delta\Delta A_1$ and $\Delta\Delta A_2$ are not state functions and their value depends on the parameterization of the Hamiltonian with respect to $\lambda$. Different parameterizations will give different decompositions, even when the initial and final states are the same, which makes interpretation of such decompositions challenging.



\subsection{The free energy can be expressed as a cumulant expansion.}

Eq.~\ref{eq:dA} can be re-written as an ensemble average over state $\lambda$
\begin{equation}
\Delta A(\lambda) =
	\beta^{-1} \ln \left\langle e^{\beta H(\lambda)} \right\rangle_\lambda.
\label{eq:avg_expr}
\end{equation}
We will relate this expression to the cumulant generating function for the random variable $\beta H$, defined as:
\begin{align}
K_{\lambda}(t) =&
	\ln \left\langle 
    	e^{t \beta H(\lambda)}
    \right\rangle_\lambda \\
    =& 
    \sum_{n=1}^{\infty}
            	\frac{t^n}{n!}
                \left[ \frac{d^n}{dt^n} K_{\lambda}\right]_{t=0},
\end{align}
where we have introduced the auxiliary variable $t$ and Maclaurin series expanded around $t=0$.

The free energy is related to the cumulant generating function by:
\begin{align}
\Delta A(\lambda) =& \beta^{-1} K_{\lambda}(1) \nonumber\\
                  =& \beta^{-1} \sum_{n=1}^{\infty}
            			\frac{1}{n!}\left[
                        	\frac{d^n}{dt^n} K_{\lambda}
                        \right]_{t=0}.
\label{eq:cumu_dA}
\end{align}

The derivatives of $K_{\lambda}$ evaluated at $t=0$ are called cumulants. The first few examples are:
\begin{align}
\kappa_1 &=
	\left[\frac{d}{dt} K_{\lambda}\right]_{t=0} =
	\beta \left\langle H \right\rangle_\lambda \nonumber\\
\kappa_2 &=
	\left[\frac{d^2}{dt^2} K_{\lambda}\right]_{t=0} =
	\beta^2 \left[
		\left\langle H^2 \right\rangle_\lambda -
    	\left\langle H \right\rangle_\lambda^2
    \right] \nonumber\\
\kappa_3 &=
	\left[\frac{d^3}{dt^3} K_{\lambda} \right]_{t=0} =
	\beta^3 \left[
		\left\langle H^3 \right\rangle_\lambda -
    	3 \left\langle H^2 \right\rangle_\lambda
    		\left\langle H \right\rangle_\lambda +
    	2 \left\langle H \right\rangle_\lambda^3
    \right].
\label{eq:cumu}
\end{align}

Up to third order, the cumulants are equal to the corresponding central moments. For fourth and higher-order cumulants, this is no longer true and the relationship is more complicated, but the cumulants to a given order can be always be expressed as a polynomial of the moments up to the same order.





\subsection{The cumulants represent the enthalpic and entropic components of the free energy.}

\begin{figure}[htb]
\centering
\includegraphics[width=3in]{figure1.pdf}
\caption{Graphical breakdown of the free energy in terms of cumulants.}
\label{fig:1D}
\end{figure}

We can recast Eq.~\ref{eq:cumu_dA} graphically as a sum over an infinite 1-dimensional array of cells (Figure~\ref{fig:1D}), where the $n$th cell contains the $n$th cumulant along with the factor $\beta^{-1}/n!$. The first cell ($n=0$) is always zero. The next cell ($n=1$) is the enthalpy, $\langle H \rangle_\lambda$. The sum of the remaining cells ($n=2...$) is the entropic part of the free energy.

This breakdown makes sense, as the enthalpy is simply the average energy of the system, which is exactly the first cumulant. Meanwhile, the entropic part of the free energy depends on the ``peakedness'' or ``flatness'' of the ensemble, which is exactly what is characterized by the the higher-order cumulants.

% Several properties of cumulants have important statistical mechanical implications. First, the cumulants are semi-invariant, meaning that $\kappa_1(X+c)=\kappa_1(X)+c$ and $\kappa_n(X+c) = \kappa_n(X)$ for some constant $c$ and $n \ge 2$. In statistical mechanics, this implies that if the energy of every microstate in a system is changed by $c$, the enthalpy also increases by $c$, whereas the entropy is unchanged. Second, the cumulants are additive $\kappa_n(X + Y) = \kappa_n(X) + \kappa_n(Y)$ when $X$ and $Y$ are independent. In statistical mechanics, this implies that for a composite system composed of independent sub-systems, the the total enthalpy or entropy is just the sum of the sub-system enthalpies or entropies.

\subsection{The free energy can be decomposed using a multivariate cumulant expansion.}

We can also relate Eq.~\ref{eq:avg_expr} to the multivariate cumulant generating function for the random variables $\beta H_1$ and $\beta H_2$, defined as:
\begin{equation}
K_\lambda(\vec t) =
	\ln \left\langle 
    	e^{t_1 \beta H_1(\lambda) + t_2 \beta H_2(\lambda)}
    \right\rangle_\lambda.
\end{equation}
Series expansion around $\vec t = (t_1, t_2) = 0$ gives:
\begin{align}
\Delta A(\lambda) =& \beta^{-1} K_\lambda(1, 1) \nonumber\\
                  =& \beta^{-1} \sum_{n=1}^{\infty}
	        			\sum_{i+j=n}
            			\frac{1}{i!j!}\left[ D_{i,j} K_\lambda\right]_{\vec t=0},
\label{eq:mult_expansion}
\end{align}
where the notation $D_{i,j}$ means $d^{i+j} /(d t_1^i d t_2^j)$. This expansion readily generalizes to an arbitrary number of Hamiltonian components. 

The mixed derivatives with $i>0, j>0$ give the joint cumulants, See Appendix 1 for a discussion of their intuitive meaning.




\subsection{The free energy can be decomposed into terms due to enthalpy, marginal entropy, and mutual information.}

\begin{figure}[htb]
\centering
\includegraphics[width=3in]{figure2.pdf}
\caption{Graphical breakdown of free energy in terms of joint cumulants.}
\label{fig:2D}
\end{figure}

Eq.~\ref{eq:mult_expansion} can be also be recast graphically, but this time as the sum of an infinite 2-dimensional array (Figure~\ref{fig:2D}). The $i,j=(0,0)$ cell is always zero. The $(1,0)$ and $(0,1)$ cells are the enthalpic components $\langle H1 \rangle_\lambda$ and $\langle H_2 \rangle_\lambda$, respectively. The sum of the remaining terms is the entropic component of the free energy. We note that cell $n$ in Figure~\ref{fig:1D} is the sum of the cells where $i+j=n$ in Figure~\ref{fig:2D}. 

The entropic contribution can further be decomposed into two marginal entropic contributions, $-\beta^{-1}S(H_1)$ and $-\beta^{-1}S(H_2)$, and a contribution due to the mutual information $\beta^{-1}I(H_1; H_2)$, with:
\begin{align}
S(X) =& \int p(X) \ln p(X) dX \nonumber\\
I(X; Y) =& \int p(X,Y) \ln \frac{p(X,Y)}{p(X)p(Y)} dX dY \nonumber\\
p(X) =& \int p(X,Y) dY.
\end{align}
Clearly, deciding how to treat the terms due to the mutual information is one of the main decisions that must be made in performing a free energy decomposition.

This graphical decomposition extends to higher dimensions, although it becomes difficult to visualize. Figure~\ref{fig} shows the case for three Hamiltonian components. There are three enthalpies: $\langle H_1 \rangle_\lambda$, $\langle H_2 \rangle_\lambda$, $\langle H_3 \rangle_\lambda$, three marginal entropies: $S(H_1)$, $S(H_2)$, $S(H_3)$, three two-body mutual information terms: $I(H_1; H_2)$, $I(H_1; H_3)$, $I(H_2; H_3)$, and a single three-body mutual information term: $I(H_1; H_2; H_3)$, with:
\begin{equation}
I(X; Y; Z) = \int p(X,Y,Z) \ln \frac{p(X,Y)p(X,Z)p(Y,Z)}{p(X,Y,Z)p(X)p(Y)p(Z)} dX dY dZ.
\end{equation} 






\subsection{Splitting the cumulants results in a path-independent free energy decomposition.}

The cumulants are state functions, so we can define path-independent free energy decompositions by splitting the cumulants between the different components:
\begin{equation}
\Delta A^{(u)}(\lambda) =
	\beta^{-1} \sum_{n=1}^{\infty}
	\sum_{i+j=n}
	\frac{a_{i,j}^{(u)}}{i!j!}\left[ D_{i,j} K_\lambda\right]_{t=0},
\end{equation}
where $u=1, 2, \ldots$ indexes the components, and $a_{i,j}^{(u)}$ are the splitting coefficients. The splitting coefficients divide each cumulant between the free energy components and are subject to $0 \le a_{i,j}^{(u)} \le 1$ and $\sum_u a_{i,j}^{(u)}=1$.

The free energy components can be evaluated by thermodynamic integration:
\begin{align}
\Delta\Delta A^{(u)} =& \Delta A^{(u)}(1) - \Delta A^{(u)}(0) \nonumber \\
					 =&
	\beta^{-1} \int_0^1 \sum_{n=1}^{\infty}
	\sum_{i+j=n}
	\frac{a_{i,j}^{(u)}}{i!j!}
    \left[ D_{i,j} \frac{dK_\lambda}{d\lambda}\right]_{t=0} d\lambda
    \label{eq:split}.
\end{align}
Only in a few cases do most of terms in Eq.~\ref{eq:split} cancel. In the general case, an infinite number of terms are present. However, we show below that we are able to partially correct a path-dependent free energy decomposition by correctly splitting the low order cumulants, while leaving a path-dependent splitting of the higher-order terms.

Evaluating the derivative of the cumulant generating function with respect to $\lambda$ gives:
\begin{equation}
K_\lambda' = 
\frac{dK_\lambda}{d\lambda} =
	\beta t_1 \left\langle H_1' \right\rangle_{\vec t,\lambda} -
    \beta \left\langle H_1' \right\rangle_{\vec t,\lambda} +
    \beta t_2 \left\langle H_2' \right\rangle_{\vec t,\lambda} -
    \beta \left\langle H_2' \right\rangle_{\vec t,\lambda}
\end{equation}
where we have dropped terms with no $\vec t$-dependence that will disappear in subsequent steps, the prime indicates the derivative with respect to $\lambda$, and the notation $\langle X \rangle_{\vec t, \lambda}$ means
\begin{equation}
\langle X \rangle_{\vec t, \lambda}  =
	\frac
    	{\int X(x, \lambda) 
        	\exp\left[
        		\beta(t_1-1)H_1(\lambda) +
            \beta(t_2-1)H_2(\lambda)
        \right] dx
        }
    	{\int
        	\exp\left[
            \beta(t_1-1)H_1(\lambda) +
            \beta(t_2-1)H_2(\lambda)
        \right] dx
        }.
\end{equation}

The general form of $D_{i,j}K'$ is:
\begin{equation}
[D_{i,j}K_\lambda']_{\vec t=0} =
	\beta\left[
		i \phi_{i-1, j}^{(1)}(\lambda) -
    	\phi_{i,j}^{(1)}(\lambda) +
    	j \phi_{i, j-1}^{(2)}(\lambda) -
    	\phi_{i,j}^{(2)}(\lambda)
    \right],
\label{eq:deriv}
\end{equation}
where we have introduced the notation
\begin{equation}
\phi_{i,j}^{(u)}(\lambda) =
	\left[ D_{i,j} \left\langle
    	H_u'
    \right\rangle_{\vec t, \lambda} \right]_{\vec t=0}.
\end{equation}
Each term $\phi_{i,j}^{(1)}$ can be produced in two different ways, from $D_{i,j}K'$ or $D_{i+1,j}K'$. Similarly, each $\phi_{i,j}^{(2)}$ can be produced from $D_{i,j}K'$ or $D_{i,j+1}K'$.

Regrouping similar terms, we can rewrite Eq.~\ref{eq:split} as:
\begin{equation}
\Delta\Delta A^{(u)} =
	\int_0^1 \left(
        a_{1,0}^{(u)}\left\langle H_1' \right\rangle +
        a_{0,1}^{(u)}\left\langle H_2' \right\rangle +
        \psi_1^{(u)}(\lambda) +
        \psi_2^{(u)}(\lambda)
    \right) d\lambda,
\end{equation}
with
\begin{align}
\psi_1^{(u)}(\lambda) &=
%	\sum_{n=1}^{\infty}
%    \sum_{i+j=n}
%        \phi_{i,j}^{(1)}\left(
%            \frac
%                {(i+1)a_{i+1,j}^{(u)}}
%                {(i+1)!j!} -
%            \frac
%                {a_{i,j}^{(u)}}
%                {i!j!}
%        \right) \nonumber\\
%    &=
	\sum_{n=1}^{\infty}
    \sum_{i+j=n}
        \frac{\phi_{i,j}^{(1)}}{i!j!}
        \left(
            {a_{i+1,j}^{(u)}} -
            {a_{i,j}^{(u)}}
        \right) \label{eq:psi1}\\
\psi_2^{(u)}(\lambda) &=
%	\sum_{n=1}^{\infty}
%    \sum_{i+j=n}
%        \phi_{i,j}^{(2)}\left(
%            \frac
%                {(j+1)a_{i,j+1,}^{(u)}}
%                {i!(j+1)!} -
%            \frac
%                {a_{i,j}^{(u)}}
%                {i!j!}
%      	\right)\nonumber\\
%    &=
	\sum_{n=1}^{\infty}
    \sum_{i+j=n}
        \frac{\phi_{i,j}^{(2)}}{i!j!k!}
        \left(
            a_{i,j+1}^{(u)} -
            a_{i,j}^{(u)}
      	\right)\label{eq:psi2}.
\end{align}



\subsection{Most terms cancel in the expression for the total free energy.}

Consider the case where there is only a single free energy component, i.e. the total free energy. In this case, the splitting coefficient is always unity, $a_{i,j}^\mathrm{total}=1$. Under these conditions, the bracketed terms in Eqs.~\ref{eq:psi1} and \ref{eq:psi2} evaluate to zero and the free energy is:
\begin{equation}
\Delta\Delta A = \int_0^1 \bigg(
	\left\langle H_1' \right\rangle +
    \left\langle H_2' \right\rangle
\bigg) d\lambda,
\end{equation}
which is equivalent to Eq.~\ref{eq:TI} as expected.





\subsection{There are no non-trivial splittings for arbitrary $\lambda$-dependence.}

Consider an arbitrary $\lambda$-dependence of the Hamiltonian. Cancellation of the bracketed terms in Eqs.~\ref{eq:psi1} and \ref{eq:psi2} requires that
\begin{align*}
a_{i+1,j}^{(u)} &= a_{i,j}^{(u)} \\
a_{i,j+1}^{(u)} &= a_{i,j}^{(u)},
\end{align*}
which has only trivial solutions like $a_{i,j}^{(1)} = a_{i,j}^{(2)} = 1/2$ that partition the free energy into fixed fractions of the total.

In Appendix 2, we show that the particular case of an identical scaling, $H(\lambda) = f(\lambda)h_1 + f(\lambda)h_2$, results in a well-defined splitting of the cumulants with nearly complete cancelation of terms, in agreement with previous results.

Although there is no splitting that results in cancellation for arbitrary $\lambda$-dependence, we will next show that it is possible to partially correct a path-dependent decomposition towards path-independence.





\subsection{The path-dependence of free energy decompositions can be partially eliminated by performing a correction to finite order.}

Consider the sum of terms due to $\psi_1$:
\begin{align*}
\sum_u \psi_1^{(u)} =&
	\sum_{n=1}^{\infty}
    \sum_{i+j=n}
        \frac{\phi_{i,j}^{(1)}}{i!j!}
        \left(
        		\sum_u a_{i+1,j}^{(u)} -
        		\sum_u a_{i,j}^{(u)}
        \right) = 0,
\end{align*}
where the second equality holds because each set of splitting coefficients sums to unity. A similar result holds for the terms due to $\psi_2$.

We can view each order $n$ of Eqs.~\ref{eq:psi1} and \ref{eq:psi2} as a correction towards a particular path-independent splitting. Each order of correction leaves the total free energy invariant, but shifts some free energy between the components. Thus, the path-dependence can be partially corrected by truncating Eqs.~\ref{eq:psi1} and \ref{eq:psi2} at finite order without affecting the total free energy.










\section{Numerical Results}

\subsection{Corrections to first order are sufficient to give nearly path-independent free energy decompositions for a simple test system.}

\subsection{An alternative decomposition provides an estimate of free energy change in the absence of Hamiltonian components.}

\section{Conclusions}

\section*{Appendices}
\subsection*{Appendix 1. The joint cumulants capture the statistical dependence of Hamiltonian components.}

The mixed partial derivatives with respect to $\vec t$ capture the statistical dependence of Hamiltonian components to a given order. For example, the second order $\kappa_{0,1,1}$ term is simply the covariance between $H_1$ and $H_2$:
\begin{align}
\kappa_{0, 1, 1} = [D_{0, 1, 1} K_\lambda]_{\vec t=0} =&
	\beta^2 \left[
    	\left\langle H_1 H_2 \right\rangle_\lambda -
		\left\langle H_1 \right\rangle_\lambda 
		\left\langle H_2 \right\rangle_\lambda
    \right] \nonumber\\
    =&
    \beta^2 \mathrm{Cov}(H_1,H_2).              
\end{align}
This cumulant the difference between the actual average value of the product of $H_1$ and $H_2$ compared to that expected if $H_1$ and $H_2$ were statistically independent.

Similarly, the third order cumulant, $\kappa_{1,1,1}$, captures the difference between the average product of $H_0$, $H_1$, and $H_2$ compared to what is expected if they were statistically independent:
\begin{alignat}{3}
\MoveEqLeft[3] [D_{1, 1, 1} K_\lambda]_{\vec t=0}\notag\\
&= \beta^3[ &&
\left\langle H_0 H_1 H_2 \right\rangle_\lambda -
\left\langle H_0 H_1 \right\rangle_\lambda
	\left\langle H_2 \right\rangle_\lambda -
\left\langle H_0 H_2 \right\rangle_\lambda
	\left\langle H_1 \right\rangle_\lambda - \notag\\
&&& \left\langle H_1 H_2 \right\rangle_\lambda
	\left\langle H_0 \right\rangle_\lambda +
2 \left\langle H_0 \right\rangle_\lambda
	\left\langle H_1 \right\rangle_\lambda
	\left\langle H_2 \right\rangle_\lambda]\notag\\
&= \beta^3[ &&
	\left\langle H_0 H_1 H_2 \right\rangle_\lambda -
	\mathrm{Cov}(H_0,H_1) \left\langle H_2 \right\rangle_\lambda -
	\mathrm{Cov}(H_0,H_2) \left\langle H_1 \right\rangle_\lambda -\\
	&&& \mathrm{Cov}(H_1,H_2) \left\langle H_0 \right\rangle_\lambda -
	\left\langle H_0 \right\rangle_\lambda
		\left\langle H_1 \right\rangle_\lambda
		\left\langle H_2 \right\rangle_\lambda]        
\end{alignat}
This cumulant accounts for all of the ways that there can be independence, where either all three variables being independent, or where two of the variables are dependent, but the third is independent.

Expressions for higher-order cumulants rapidly become more complex, but they all capture the same...





\subsection{Appendix 2. Identical scaling leads to a natural splitting where most terms cancel.}

Consider the case where the $\lambda$-dependence is an identical scaling, $H(\lambda) = f(\lambda)h_1 + f(\lambda)h_2$. In these circumstances,
\begin{equation*}
\frac{\partial}{\partial t_1}
	\langle H_2' \rangle_{\vec t, \lambda} = 
\frac{\partial}{\partial t_2}
	\langle H_1' \rangle_{\vec t, \lambda},
\end{equation*}
which implies
\begin{equation}
\phi_{i,j}^{(2)} = \phi_{i-1,j+1}^{(1)}.
\label{eq:separable}
\end{equation}
Substitution of Eq.~\ref{eq:separable} into Eq.~\ref{eq:deriv} gives
\begin{align}
[D_{i,j}K_\lambda']_{\vec t=0} &=
	\beta\left[
		i \phi_{i-1, j}^{(1)}(\lambda) -
    	\phi_{i,j}^{(1)}(\lambda) +
    	j \phi_{i-1, j}^{(1)}(\lambda) -
    	\phi_{i-1,j+1}^{(1)}(\lambda)
    \right] \nonumber \\
    &=
	\beta\left[
		(i + j)\phi_{i-1, j}^{(1)}(\lambda) -
    	\phi_{i,j}^{(1)}(\lambda) -
    	\phi_{i-1,j+1}^{(1)}(\lambda)
    \right].
\end{align}
Each term $\phi_{i,j}^{(1)}$ can be produced in three different ways, from $D_{i,j}K'$, $D_{i+1,j}K'$, or $D_{i+1,j-1}K'$. Grouping of similar terms gives
\begin{equation}
\Delta\Delta A^{(u)} =
	\int_0^1 \left(
        a_{1,0}^{(u)}f'(\lambda)
        \left\langle h_1 \right\rangle +
        a_{0,1}^{(u)}f'(\lambda)
        \left\langle h_2 \right\rangle +
        \sigma(\lambda)
    \right) d\lambda,
\end{equation}
with
\begin{align}
\sigma(\lambda) &=
	\sum_{n=1}^{\infty}
    \sum_{i+j=n}
        \frac
        	{\phi_{i,j}^{(1)}}
            {i!j!}
        \left(
            \frac
                {(i+j+1)a_{i+1,j}^{(u)}}
                {i+1} -
            a_{i,j}^{(u)} -
           	\frac
            	{j a_{i+1,j-1}^{(u)}}
                {i+1}
		\right).
\label{eq:cancel}
\end{align}
Cancellation of the bracketed terms in Eq.~\ref{eq:cancel} requires
\begin{equation}
\frac
	{(i+j+1)a_{i+1,j}^{(u)}}
	{i+1} -
a_{i,j}^{(u)} -
\frac
	{j a_{i+1,j-1}^{(u)}}
	{i+1}
= 0.
\end{equation}
Under the constraints $a_{1,0}^{(1)} = 1$ and $a_{0,1}^{(2)} = 1$, the solution is
\begin{align}
a_{i,j}^{(1)} &= \frac{i}{i+j} \nonumber\\
a_{i,j}^{(2)} &= \frac{j}{i+j} \nonumber\\
\label{eq:splitting}
\end{align}
leading to
\begin{equation}
\begin{split}
\Delta\Delta A^{(1)} &= 
	\int_0^1 f'(\lambda)
    \langle h_1 \rangle 
    d\lambda \\
\Delta\Delta A^{(2)} &= 
	\int_0^1 f'(\lambda)
    \langle h_2 \rangle
    d\lambda.
\end{split}
\end{equation}
Thus, in the case of identical scaling, the na\"ive scaling given by Eq.~\ref{eq:naive} results in a well-defined splitting of the cumulants. This result is similar to that in previous work.


\end{document}